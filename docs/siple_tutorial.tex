\documentclass[12pt]{article}
\usepackage[lf]{MinionPro}
\usepackage{amsmath}
\usepackage[left=1in,right=1in,top=1in,bottom=1in]{geometry}
\def\siple{{\sf siple}}
\newcommand{\calF}{\mathcal{F}}
\newcommand{\Reals}{\mathbb{R}}
\DeclareMathOperator{\Laplacian}{\Delta}
\let\ra\rightarrow
\title{\siple: A Small Inverse Problems Library}
\author{David Maxwell}
\begin{document}
\maketitle

\section{Inverse Problems Overview}

Inverse problems are, as the language suggests, the inverses of forward problems. For us, a forward problem will be a function $\calF$ 
from a space $X$ to a space $Y$, and we will insist that this function be continuous.  That is, $X$ and $Y$ will be equipped with a notion of distance, and if $\hat x$ is brought near $x$, then $\hat y=\calF(\hat x)$ can be made as close to $y=\calF(y)$ as we wish.  

For example, let $\rho$ be a density of matter in space $\Reals^3$ that
is contained inside some ball of radius $R$ about the origin. The forward 
problem is to find the gravitational potential of $\rho$.  That is, find
a function $\Phi=\calF(\rho)$ solving
\begin{equation}
-\Laplacian \Phi = 4\pi G \rho
\end{equation}
where $G$ is the universal gravitational constant.  For the problem
to be well-defined, we need to add a boundary condition, and we
require the decay condition that $\Phi\ra 0$ at infinity.  

Showing this  particular problem is continuous requires some technical details: a good choice of distance is needed for the spaces $X$ and $Y$ of matter distributions and gravitational potentials (so-called weighted $L^2$ spaces would be reasonable).  But the idea that small perturbations in $\rho$ make correspondingly small perturbations in $\Phi$ is clear.  The moon doesn't care when I walk around on the Earth.

The corresponding inverse problem is as follows.  Given a potential field $\Phi$, determine $\rho$.  This problem has a very different character than the original problem.  If you perturb $\Phi$ in a way so that its second derivatives don't change much, then you won't change $\rho$ much.  
But measurements of physical fields frequently measure their values but not their derivatives.  A measurement $\hat \Phi$ of $\Phi$ might have values that are close (in some sense) to $\Phi$ but give no control on its derivatives.
For example, $\hat \Phi$ could be obtained from $\Phi$ by adding a tiny but very high-frequency oscillation.  This difficulty is a typical feature of a true inverse problem.

In general terms, the inverse problem for $\calF$ is as follows.  Given $y\in Y$, find $x\in X$ such that
\begin{equation}\label{eq:inverse}
\calF(x) = y.
\end{equation}
The theory of inverse problems focusses on problems where the inverse problem is ill-posed.  That is, at least one of the following fails:
\begin{enumerate} 
    \item Given any $y\in Y$, there is a solution $x\in X$ of equation \eqref{eq:inverse}.
    \item Given any $y\in Y$, there is no more than one solution $x\in X$ of equation \eqref{eq:inverse}.
    \item The solution of \eqref{eq:inverse} depends continuously on on $y\in Y$.
\end{enumerate}
The difficulties posed by a failure of requirements 1) and 2) are not the focus of \siple, and can sometimes be dealt with by expanding the notion of a solution or specification of additional constraints.  This was done, for example, in the forward problem for the gravitational potential where the constraint that $\Phi\ra 0$ at infinity was added.

Requirement 3) is more subtle and shows up frequently in cases (like the density problem) where the forward problem involves solving a differential equation, and the inverse problem involves taking a derivative.  It is 
typically dealt with by so-called regularization methods.

\subsection{Regularization}

Suppose $\calF(x)=y$ and $y^\delta$ is an approximation of $y$.



\section{The derivative problem}

We will work with the following model problem.  Given measurements
of a function $f$ defined on a circle, determine its derivative.



\end{document}